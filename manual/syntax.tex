\section{Syntax}
\begin{normalsize}
CQL is designed to replicate the expressive power of conjunctive queries, and as such the syntax can largely be mirrored in the language of \textit{conjunctive queries}. In this section we will expand on this relationship, in addition to exploring the syntax and briefly looking at some programmer friendly features within the syntax.

\subsection{Commenting}

Commenting is supported within CQL, with syntactic sugar offered with the dual style. Both \codeword{//} and \codeword{#} are acceptable commenting styles.
\begin{lstlisting}[style=framed]
// This is a comment
This is not

#This is also a comment
This is not
\end{lstlisting}
CQL also supports in line commenting. For example:
\begin{lstlisting}[style=framed]
1,2,3 where A(1,2) and B(2,3) // This is an end of line comment
1,2 where P(1) and Q(2) and 1 = 2  # This is also an end of line comment.
\end{lstlisting}


\subsection{Formula Judgements}
Judgements are assertions in mathematical logic, in this case the occurrence of a free variable in an expression. \textit{Conjunctive queries} use formula judgements to list all free variables in an expression E, formally defined as:
$$ \overrightarrow{x} \vdash \varphi $$
Where $\overrightarrow{x}$ is a list of all free variables used in a program $\varphi$.  In CQL, this syntax is replicated with:
\begin{lstlisting}[style=framed]
x,y where P(x) and Q(y) = z
\end{lstlisting}
Where $x$, $y$ are free variables in the expression \lstinline!P(x) and Q(y)!. Note that equalities cannot reference free variables. For example, given the above example, $z$ cannot be included as a free variable (for example: \lstinline!x,y,z where P(x) and Q(y) = z!). All not declared 
\end{normalsize}

\subsection{Existential Qualification}
Existential qualification is logically equivalent to 'there exists', denoted $\exists$ in \textbf{conjunctive queries}. In CQL, this relationship is implied, and all free variables not declared in Formula Judgements will be not be outputted. Formally:
$$ x_1 \vdash \exists z.R(x,z) $$
This, written in CQL, is equivalent to:
\begin{lstlisting}[style=framed]
x where R(x,z)
\end{lstlisting}
Where z is declared existentially, i.e. 'There exists' some value of z, the value of which we don't care about. It is also important to note that we can combine existential qualification with logical conjunction (AND), however we will address this in section \textbf{TODO}


\subsection{Logical conjunction}
Logical conjunction a is provided within CQL using the \codeword{and} keyword. Simply, logical conjunction can be applied between relational symbols. Formally:
$$x_1, x_2, x_3, x_4 \vdash A(x_1, x_2)  \land B(x_3, x_4)$$

\begin{lstlisting}
1,2 where A(1,3) and B(3,2)
\end{lstlisting}

In this statement, 1 and 2 are declared, and 3 forms an existential qualification.

