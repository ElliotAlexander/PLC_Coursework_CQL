\section{Syntax}
\begin{normalsize}
CQL is designed to replicate the expressive power of conjunctive queries, and as such the syntax can largely be mirrored in the language of \textit{conjunctive queries}. In this section we will expand on this relationship, in addition to exploring the syntax and briefly looking at some programmer friendly features within the syntax.

\subsection{Commenting}

Commenting is supported within CQL, with syntactic sugar offered with the dual style. Both \codeword{//} and \codeword{#} are acceptable commenting styles.
\begin{lstlisting}[style=framed]
// This is a comment
This is not

#This is also a comment
This is not
\end{lstlisting}
CQL also supports in line commenting. For example:
\begin{lstlisting}[style=framed]
1,2,3 where A(1,2) and B(2,3) // This is an end of line comment
1,2 where P(1) and Q(2) and 1 = 2  # This is also an end of line comment.
\end{lstlisting}
\end{normalsize}